% %\documentclass[a4paper,10pt]{article}
% \documentclass[tikz]{standalone}
% \usetikzlibrary{shapes,arrows,fit,calc,positioning}
% \usepackage{amsmath, amsthm, amssymb,latexsym,amscd,amsfonts,enumerate}
% \usepackage[utf8]{vietnam}
% \usepackage{subfigure}
% \usepackage{secdot}
% \usepackage{graphicx}
% \usepackage{booktabs}
% \usepackage{pgfgantt}
% \usepackage{tabularx}
% \usepackage{ltablex}
% \usepackage{amsthm}
% \usepackage{amsmath}
% \usepackage{amsfonts}
% \usepackage{amssymb}
% \usepackage{graphicx} 
% \usepackage{titling}
% \usepackage{secdot}
% \usepackage{enumitem}
% \usepackage{tikz}
% \usepackage{array}
% \usetikzlibrary{calc}
% \usepackage{longtable}
% \usepackage{indentfirst}
% \usepackage{fancyhdr}
% \usepackage{exscale,relsize,makeidx}
% \usepackage{color, fancyhdr, graphicx, wrapfig}
% \begin{document}

% \begin{table}
%     \begin{tabular}{| c | c | c |} 
%     \hline
%     Job ($j$) & $p_j$ & $w_j$ \\
%     \hline
%     1 & $p_1$ & $w_1$ \\
%     2 & $p_2$ & $w_2$ \\
%     $\vdots$ & $\vdots$ & $\vdots$ \\
%     $k$ & $p_k$ & $w_k$ \\
%     $\vdots$ & $\vdots$ & $\vdots$ \\
%     $n$ & $p_n$ & $w_n$ \\
%     \hline
%     \end{tabular}
% \end{table}
% Giả sử $p_k = C$ với $C$ là một số vô cùng lớn $(C= \infty)$.

% \begin{equation}
% \frac{\sum_{j=1}^{n}w_j}{\sum_{j=1}^{k-1}p_j + C}
% \end{equation}



% \end{document}
% You need the tree-dvips package to process this file.
% Alexis Dimitriadis (alexis@ling.upenn.edu) 10/10/99

\documentclass{article}

\advance\textheight by 1in
\advance\textwidth by 0.5in
\topmargin=-0.3in

\usepackage{tree-dvips}
\usepackage{qtree}

% PostScript to PDF conversion using ghostscript (alias ps2pdf) 
% looks better on-screen if one of the built-in PDF fonts is used:
\usepackage{times} 

\begin{document}
\centerline{\large\bf Examples: Drawing arrows on \emph{qtree} trees}
\medskip
The arrow-drawing capabilities of the package \emph{tree-dvips} (written by
Emma Pease) can be used with trees drawn with \emph{qtree}.  The two packages
are fully compatible.  

Note, however, that tree-dvips relies on PostScript specials, and thus does not work with pdf\LaTeX. This file was generated as DVI and then converted to pdf. 

\emph{Tree-dvips} is not included in the distribution of \emph{qtree;} it is
available on CTAN.

Thanks to Seth Kulick for telling me about the combination, and to Amanda
Seidl for contributing the verb-movement example. 
\bigskip

\hskip 1.5cm\Tree [ [ \node{subj1}subj_i ].NP [  [
 T+v_n+\node{V}V_j+Apl_k
 ].T [ \node{io}{ }IO_l
 [ \node{subj2}t_i [ \node{v1}t_n  [ \node{do}DO_m  [  \node{io1}t_l
  [ \node{apl1}t_k [   [  \node{V1}t_j  ].V
 \node{do1}t_m  ].VP ].Apl\1
 ].Apl\1 ].AplP ].{\it v}\1 ].{\it v}\1  ].{\it v}P ].T\1  ].TP

 \anodecurve[bl]{subj2}[bl]{subj1}{0.4in}%
 \anodecurve[bl]{do1}[bl]{do}{0.4in}%
 {\makedash{4pt}\anodecurve[t]{io1}[r]{io}{.5in}}%
 \anodecurve[bl]{V1}[bl]{apl1}{0.6in}%
 \anodecurve[bl]{apl1}[bl]{v1}{1in}%
 \anodecurve[bl]{v1}[bl]{V}{0.9in}%

% These would give square movement arrows instead:
%
%  \abarnodeconnect[-6pt]{subj2}{subj1}{0.4in}
%  \abarnodeconnect[-6pt]{do1}{do}{0.4in}
%  {\makedash{4pt}{\anodecurve[t]{io1}[r]{io}{.5in}}}
%  \abarnodeconnect[-6pt]{V1}{apl1}{0.6in}
%  \abarnodeconnect[-6pt]{apl1}{v1}{1in}
%  \abarnodeconnect[-6pt]{v1}{V}{0.9in}

\vspace*{-0.95in}
\noindent
{\small\begin{verbatim}
\Tree 
[ [ \node{subj1}subj_i ].NP 
  [  [ T+v_n+\node{V}V_j+Apl_k ].T 
       [ \node{io}{ }IO_l
         [ \node{subj2}t_i [ \node{v1}t_n  
             [ \node{do}DO_m  [  \node{io1}t_l
                 [ \node{apl1}t_k [ [ \node{V1}t_j ].V
                   \node{do1}t_m  ].VP ].Apl\1	
           ].Apl\1 ].AplP ].{\it v}\1 ].{\it v}\1  
  ].{\it v}P ].T\1  ].TP

\anodecurve[bl]{subj2}[bl]{subj1}{0.4in}
\anodecurve[bl]{do1}[bl]{do}{0.4in}
{\makedash{4pt}\anodecurve[t]{io1}[r]{io}{.5in}}
\anodecurve[bl]{V1}[bl]{apl1}{0.6in}
\anodecurve[bl]{apl1}[bl]{v1}{1in}
\anodecurve[bl]{v1}[bl]{V}{0.9in}
\end{verbatim}}

% A little bug: Because of how TeX assigns type categories to its input, the
% automatic math-mode switching of label subscripts does not work inside
% footnotes or boxes.  To work around this, invoke \verb|\automath|
% \emph{before} entering in the box or footnote.  (This will enable
% auto-switching to math mode anywhere in the text).
\automath

\noindent\hskip-0.3cm
\parbox{1.2in}{%
\Tree [.S [.NP \node{subj}{subj_i} ] 
          [.VP [.V verb ] [.NP \node{t}{t_i} ]]] \bigskip}
\abarnodeconnect[-6pt]{t}{subj}
% Another problem: \verbatim cannot be used inside anything boxes or 
% footnotes (for the same reason).  Use this substitute.
{\obeyspaces\obeylines%
\parbox{3.9in}{\tt\chardef\\=`\\%         % Let \\ be \backslash
\\Tree [.S [.NP \\node\{subj\}\{subj\_i\} ]  
\          [.VP [.V verb ] [.NP \\node\{t\}\{t\_i\} ]]]
~
\\abarnodeconnect[-6pt]\{t\}\{subj\}}}

\end{document}