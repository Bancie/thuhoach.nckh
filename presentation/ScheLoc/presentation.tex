\documentclass[10pt]{beamer}
\mode<presentation>
\usepackage{beamerthemesplit}
\usepackage{graphicx}
\usepackage{booktabs}
\usepackage{amsmath}
\usepackage{textpos}
\usepackage{pgfplots}
\usepackage{tikz}
\usepackage[dvipsnames]{xcolor}
\usepackage{hyperref}
\usepackage{caption}
\usepackage{listings}


% import citation package
\usepackage{biblatex}
\addbibresource{presentation.bib}
\AtBeginBibliography{\small}

\usetikzlibrary{shapes.geometric, arrows}
\usetikzlibrary {datavisualization} 
\pgfplotsset{compat=1.18, width = 7cm}
\usetikzlibrary{patterns}
\usetheme{Ilmenau} % AnnArbor, Ilmenau, Darmstadt, Dresden, CambridgeUS, Frankfurt, Singapore
\newtheorem{dn}{Định nghĩa}[section]
\newtheorem{dl}{Định lý}[section]
\newtheorem{tc}{Tính chất}[section]
\newtheorem{hq}{Hệ quả}[section]
\newtheorem{bd}{Bổ đề}[section]
\newtheorem{md}{Mệnh đề}[section]
\newtheorem{vd}{Ví dụ}[section]
\newtheorem{nx}{Nhận xét}[section]
\newtheorem{cy}{Chú ý}[section]
\newcommand{\dom}{\text{{\rm dom}}}
\newcommand{\epi}{\text{{\rm epi}}}
\newcommand{\Min}{\text{{\rm Min}}}
\setbeamertemplate{theorems}[numbered]
\setbeamertemplate{definitions}[numbered]
\setbeamertemplate{footline}[frame number]
\usepackage{algorithm}
\usepackage{color}
\usepackage{algorithmic}
\usepackage{footmisc}
\usepackage{indentfirst} 
\usepackage{comment}
\AtBeginEnvironment{proof}{%
  \setbeamercolor{block title}{use=example text,fg=white,bg=example text.fg!75!black}
  \setbeamercolor{block body}{parent=normal text,use=block title example,bg=block title example.bg!10!bg}
}
\renewcommand{\thefootnote}{\arabic{footnote}}
\usefonttheme{professionalfonts}
\setbeamercolor{normal text}{bg=white,fg=black}
\renewcommand{\thefootnote}{\arabic{footnote}}
\beamertemplatetransparentcoveredhigh
\usetheme[progressbar=frametitle]{metropolis}
\usepackage{appendixnumberbeamer}

\usepackage[utf8]{vietnam}

\usepackage{booktabs}
\usepackage[scale=2]{ccicons}

\usepackage{pgfplots}
\usepgfplotslibrary{dateplot}

\usepackage{xspace}
\newcommand{\themename}{\textbf{\textsc{metropolis}}\xspace}
\definecolor{mSybilaRed}{HTML}{990000}

\setbeamercolor{title separator}{
  fg=mSybilaRed
}

\setbeamercolor{progress bar}{%
  fg=mSybilaRed,
  bg=mSybilaRed!90!black!30
}

\setbeamercolor{progress bar in section page}{
  use=progress bar,
  parent=progress bar
}

\setbeamercolor{alerted text}{%
  fg=mSybilaRed
}

\setbeamertemplate{footline}
{
  \leavevmode
  \hbox{
  \begin{beamercolorbox}[wd=.15\paperwidth,ht=2.25ex,dp=1ex,center]{title in head/foot}
  \end{beamercolorbox}

  \begin{beamercolorbox}[wd=.7\paperwidth,ht=2.25ex,dp=1ex,center]{author in head/foot}
    \usebeamerfont{author in head/foot}\insertshorttitle
  \end{beamercolorbox}

  \begin{beamercolorbox}[wd=.15\paperwidth,ht=2.25ex,dp=1ex,center]{title in head/foot}
    \insertframenumber{} / \inserttotalframenumber
  \end{beamercolorbox}
  }
}

\title{Tích hợp Mô hình Vị trí và Lập lịch (ScheLoc) cho Đơn Máy}


% \titlegraphic{\hfill \includegraphics[height=1cm]{tilearn.png}}

\date{\today}
\author{Hướng dẫn: TS. Lê Minh Huy}
% \institute{Sinh viên lớp: DTU1221, Khóa: 22 @ Trường Đại học Sài Gòn}

%\title{Metropolis}
\subtitle{Thực hiện: Nguyễn Chí Bằng}
% \date{\today}
%\date{}
%\author{Matthias Vogelgesang}
%\institute{Center for modern beamer themes}
%\titlegraphic{\hfill\includegraphics[height=1.5cm]{logo.pdf}}

\definecolor{codegreen}{rgb}{0,0.6,0}
\definecolor{codegray}{rgb}{0.5,0.5,0.5}
\definecolor{codepurple}{rgb}{0.58,0,0.82}
\definecolor{backcolour}{rgb}{0.95,0.95,0.92}

\lstdefinestyle{mystyle}{
    backgroundcolor=\color{backcolour},   
    commentstyle=\color{codegreen},
    keywordstyle=\color{magenta},
    numberstyle=\tiny\color{codegray},
    stringstyle=\color{codepurple},
    basicstyle=\ttfamily\footnotesize,
    breakatwhitespace=false,         
    breaklines=true,                 
    captionpos=b,                    
    keepspaces=true,                 
    numbers=left,                    
    numbersep=5pt,                  
    showspaces=false,                
    showstringspaces=false,
    showtabs=false,                  
    tabsize=2
}

\lstset{style=mystyle}

\begin{document}

\begin{frame}
  \titlepage
\end{frame}

\begin{frame}
    \frametitle{NỘI DUNG}
    \tableofcontents
\end{frame}

\section{Giới thiệu về bài toán ScheLoc}

\begin{frame}{Giới thiệu về bài toán ScheLoc}
\end{frame}

\begin{frame}{Thuật toán tìm đường đi ngắn nhất}
\end{frame}

\begin{frame}
    
\end{frame}

\begin{frame}{Mô hình bài toán ScheLoc}

\begin{figure}[h]
    \centering
    \begin{tikzpicture}
        % Nodes
        \node[draw, circle] (vi) at (0,0) {$v_i$};
        \node[draw, circle] (vj) at (6,0) {$v_j$};
        \filldraw [black] (3,0) circle (1pt);

        
        % Line with labels
        \draw (vi) -- node[above] {$\alpha l_e$} (3,0) node[below, yshift=-2mm] {$x=(e,\alpha)$} (3,0);
        \draw (3,0) -- node[above] {$(1-\alpha) l_e$} (vj);
        
        % Dotted continuation
        \draw[dotted] (-1,0) -- (vi);
        \draw[dotted] (vj) -- (7,0);
    \end{tikzpicture}
    \caption{}
\end{figure}

\end{frame}

\begin{frame}
\begin{figure}[h]
    \centering
    \begin{tikzpicture}[node distance={20mm}, thick, main/.style = {draw, circle}, dis/.style = {}] 
        \node[main] (1) {$v_1$};
        \node[main] (2) [above right of=1] {$v_2$};
        \node[main] (3) [below right of=1] {$v_3$}; 
        \node[dis] (4) [above right of=3] {};
        \node[main] (5) [above right of=4] {$v_4$}; 
        \node[main] (6) [below right of=4] {$v_5$};
        \node[main] (7) [below right of=5] {$v_6$};

        \draw[->] (1) -- node[above left] {$4$} (2);
        \draw[->] (2) -- node[above] {$3$} (5);
        \draw[->] (5) -- node[above right] {$2$} (7);
        \draw[->] (1) -- node[below left] {$3$} (3);
        \draw[->] (3) -- node[below] {$3$} (6);
        \draw[->] (6) -- node[below right] {$2$} (7);
        \draw[->] (2) -- node[above right] {$2$} (6);
    \end{tikzpicture} 
    \caption{}
\end{figure}
\end{frame}

\begin{frame}
\begin{figure}[h]
    \centering
    \begin{tikzpicture}[node distance={20mm}, thick, main/.style = {draw, circle}, dis/.style = {}] 
        \node[main, color=LimeGreen, fill=LimeGreen, very thick, text=white] (1) {$v_1$};
        \node[main] (2) [above right of=1] {$v_2$};
        \node[main, color=LimeGreen, fill=LimeGreen, very thick, text=white] (3) [below right of=1] {$v_3$}; 
        \node[dis] (4) [above right of=3] {};
        \node[main] (5) [above right of=4] {$v_4$}; 
        \node[main, color=LimeGreen, fill=LimeGreen, very thick, text=white] (6) [below right of=4] {$v_5$};
        \node[main, color=LimeGreen, fill=LimeGreen, very thick, text=white] (7) [below right of=5] {$v_6$};

        \draw[->] (1) -- node[above left] {$4$} (2);
        \draw[->] (2) -- node[above] {$3$} (5);
        \draw[->] (5) -- node[above right] {$2$} (7);
        \draw[->, color=LimeGreen, fill=LimeGreen, very thick, text=black] (1) -- node[below left] {$3$} (3);
        \draw[->, color=LimeGreen, fill=LimeGreen, very thick, text=black] (3) -- node[below] {$3$} (6);
        \draw[->, color=LimeGreen, fill=LimeGreen, very thick, text=black] (6) -- node[below right] {$2$} (7);
        \draw[->] (2) -- node[above right] {$2$} (6);
    \end{tikzpicture} 
    \caption{}
\end{figure}
\end{frame}

\begin{frame}

\begin{figure}[h]
    \centering
    \begin{tikzpicture}[node distance={20mm}, thick, main/.style = {draw, circle}, dis/.style = {}] 
        \node[main, color=Cerulean, fill=Cerulean, very thick, text=white] (1) {$v_1$};
        \node[main, color=Cerulean, fill=Cerulean, very thick, text=white] (2) [above right of=1] {$v_2$};
        \node[main] (3) [below right of=1] {$v_3$}; 
        \node[dis] (4) [above right of=3] {};
        \node[main, color=Cerulean, fill=Cerulean, very thick, text=white] (5) [above right of=4] {$v_4$}; 
        \node[main] (6) [below right of=4] {$v_5$};
        \node[main] (7) [below right of=5] {$v_6$};

        \draw[->, color=Cerulean, fill=Cerulean, very thick, text=black] (1) -- node[above left] {$4$} (2);
        \draw[->, color=Cerulean, fill=Cerulean, very thick, text=black] (2) -- node[above] {$3$} (5);
        \draw[->] (5) -- node[above right] {$2$} (7);
        \draw[->] (1) -- node[below left] {$3$} (3);
        \draw[->] (3) -- node[below] {$3$} (6);
        \draw[->] (6) -- node[below right] {$2$} (7);
        \draw[->] (2) -- node[above right] {$2$} (6);
    \end{tikzpicture} 
    \caption{}
\end{figure}
\end{frame}

\begin{frame}{Tối thiểu thời gian hoàn thành cực đại $C_{\max}$}
\end{frame}

\section{Bài toán ScheLoc cho tập hai công việc}

\begin{frame}{Cơ sở lý thuyết và ký hiệu}
\end{frame}

\begin{frame}{Trường hợp $\pi(1)=1$ và  $\pi(2) = 2$}
\end{frame}

\begin{frame}{Trường hợp $\pi(1)=2$ và  $\pi(2) = 1$}
\end{frame}

\begin{frame}{Nghiệm tối ưu của bài toán}
\end{frame}

\begin{frame}
\end{frame}

\begin{frame}[allowframebreaks, noframenumbering]
    \nocite{*}
    \printbibliography
\end{frame}


\begin{frame}
    \begin{block}{}
    \medskip
    \center{\huge \it \textcolor[rgb]{0.37, 0.150, 0.190}{Thanks for listening!}}
    \medskip
    \end{block}	
\end{frame}    
\end{document}
